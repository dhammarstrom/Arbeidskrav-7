% Options for packages loaded elsewhere
\PassOptionsToPackage{unicode}{hyperref}
\PassOptionsToPackage{hyphens}{url}
%
\documentclass[
]{article}
\usepackage{amsmath,amssymb}
\usepackage{lmodern}
\usepackage{ifxetex,ifluatex}
\ifnum 0\ifxetex 1\fi\ifluatex 1\fi=0 % if pdftex
  \usepackage[T1]{fontenc}
  \usepackage[utf8]{inputenc}
  \usepackage{textcomp} % provide euro and other symbols
\else % if luatex or xetex
  \usepackage{unicode-math}
  \defaultfontfeatures{Scale=MatchLowercase}
  \defaultfontfeatures[\rmfamily]{Ligatures=TeX,Scale=1}
\fi
% Use upquote if available, for straight quotes in verbatim environments
\IfFileExists{upquote.sty}{\usepackage{upquote}}{}
\IfFileExists{microtype.sty}{% use microtype if available
  \usepackage[]{microtype}
  \UseMicrotypeSet[protrusion]{basicmath} % disable protrusion for tt fonts
}{}
\makeatletter
\@ifundefined{KOMAClassName}{% if non-KOMA class
  \IfFileExists{parskip.sty}{%
    \usepackage{parskip}
  }{% else
    \setlength{\parindent}{0pt}
    \setlength{\parskip}{6pt plus 2pt minus 1pt}}
}{% if KOMA class
  \KOMAoptions{parskip=half}}
\makeatother
\usepackage{xcolor}
\IfFileExists{xurl.sty}{\usepackage{xurl}}{} % add URL line breaks if available
\IfFileExists{bookmark.sty}{\usepackage{bookmark}}{\usepackage{hyperref}}
\hypersetup{
  pdftitle={Study designs},
  hidelinks,
  pdfcreator={LaTeX via pandoc}}
\urlstyle{same} % disable monospaced font for URLs
\usepackage[margin=1in]{geometry}
\usepackage{graphicx}
\makeatletter
\def\maxwidth{\ifdim\Gin@nat@width>\linewidth\linewidth\else\Gin@nat@width\fi}
\def\maxheight{\ifdim\Gin@nat@height>\textheight\textheight\else\Gin@nat@height\fi}
\makeatother
% Scale images if necessary, so that they will not overflow the page
% margins by default, and it is still possible to overwrite the defaults
% using explicit options in \includegraphics[width, height, ...]{}
\setkeys{Gin}{width=\maxwidth,height=\maxheight,keepaspectratio}
% Set default figure placement to htbp
\makeatletter
\def\fps@figure{htbp}
\makeatother
\setlength{\emergencystretch}{3em} % prevent overfull lines
\providecommand{\tightlist}{%
  \setlength{\itemsep}{0pt}\setlength{\parskip}{0pt}}
\setcounter{secnumdepth}{-\maxdimen} % remove section numbering
\ifluatex
  \usepackage{selnolig}  % disable illegal ligatures
\fi

\title{Study designs}
\author{}
\date{\vspace{-2.5em}}

\begin{document}
\maketitle

\hypertarget{spuxf8rsmuxe5l}{%
\section{Spørsmål}\label{spuxf8rsmuxe5l}}

~

I denne oppgaven skal det undersøkes hvordan ulike former for trening
påvirker VO\textsubscript{2}maks for utrente. Det er valgt ut fem
artikler som blir brukt som bakgrunn for disse undersøkelsene (Chtara,
2005; Lo et al., 2011; Nybo et al., 2010; Trapp et al., 2008; Warburton
et al., 2004). Ingen av disse studiene har som hovedmål å undersøke
utelukkende hvordan VO\textsubscript{2}maks endres over en
treningsperiode, men alle studiene bruker det som en del av
testbatteriet sitt. Tre av studiene ser på ulike varianter av hvordan
trening er helsefremmende. Nybo et al.~(2010) undersøker hvordan
intervalltrening og tradisjonell trening fremmer helse. Trapp et
al.~(2008) undersøker hvilken effekt høyintensiv intervalltrening har på
fett tap og insulinverdier i hvile hos kvinner. Lo et al.~(2011)
undersøker hvordan trening og avtrening påvirker kroppssammensetning og
fysisk prestasjon i unge menn. ~De to siste studiene hadde en litt annen
tilnærming og problemstilling enn de foregående tre. Chtara et
al.~(2005) undersøkte samtidig styrke og utholdenhetstrenings effekt på
aerob prestasjon og kapasitet. Og Warburton et al.~(2004) undersøkte
blod volum ekspansjon og kardiorespiratorisk funksjon.

~

\hypertarget{alternative-forklaringer}{%
\section{Alternative forklaringer}\label{alternative-forklaringer}}

~

Selv om alle studiene har satt seg ut å teste forskjellige
problemstillinger svarer alle til en viss grad på den problemstillingen
som man prøver å besvare i denne oppgaven. Alle studiene bruker to eller
flere treningsgrupper som trener forskjellig, og tester
VO\textsubscript{2}maks før og etter, men bare Lo et al.~(2010) har det
med som en del av hypotesen, hvor man ønsker å undersøke endringen. Det
er to av artiklene som fremstiller en hypotese (Lo et al., 2011;
Warburton et al., 2004), de tre andre studiene ønsker å undersøke
effekten en intervensjon har på ulike parameter.

~

~

~

\hypertarget{metode}{%
\section{Metode}\label{metode}}

~

Nybø et al.~(2010) brukte en randomisert kontrollstudie (RCT-studie).
Gruppen som ble testet var 36 utrente menn, som ikke hadde drevet
organisert fysisk aktivitet de siste to år. De hadde en
gjennomsnittsalder på 31 år. Ingen av deltakerne røykte og alle var uten
sykdommer. Studien sier ikke noe om hvordan dette er representativt for
populasjonen. Studien bestod av 4 grupper, en gruppe som bedrev
høyintensiv intervalltrening, en som trente styrketrening, en som trente
moderat løping, og en kontrollgruppe. De tre treningsgruppene trente 2-3
ganger i uka i 12 uker, kontrollgruppa opprettholdt samme fysiske form
som ved pre test. Alle gjennomførte testbatteriet en gang før pre test,
for å bli kjent med hvordan det skulle foregå. Prestasjonstesten ble
gjennomført på en standardisert tredemølle test, måling av alle de andre
fysiologiske variablene ble gjennomført om morgenen i fastende tilstand.
Dette ble gjennomført både før intervensjonsperioden og etter. For å
sammenligne gruppene ble det brukt en to veis ANOVA test, og en veis
ANOVA for å sammenligne repeterte målinger. Signifikansnivået ble satt
til P \textless{} 0,05.

~

Trapp et al.~(2008) brukte en RCT-studie. Gruppen var ikke røykende,
inaktive, men friske kvinner mellom 18 og 30 år. Studien sier ikke noe
om hvorvidt dette var representativt for populasjonen. Studien bestod av
tre grupper, en gruppe som trente høyintensivt intervall på sykkel, en
gruppe som trente moderat med kontinuerlig arbeid, og en kontrollgruppe.
Det var ingen signifikante forskjeller mellom gruppene ved pretest. Både
de fysiologiske testene og prestasjonstesten ble gjennomført
standardisert, og det ble gjennomført pretest før intervensjonsstart og
en posttest etter. De statistiske testene som ble gjennomført var en
post hoc test, og det ble regnet ut en korrelasjonskoeffeisent.

~

Chtara et al.~(2005) brukte en RCT studie. Gruppene bestod av 48
mannlige idrettsstudenter med en gjennomsnittsalder på 21,4 år, som ikke
drev med idrett utenfor studiet. Alle hadde sett protokollen, men ingen
visste målet med studien. Studien bestod av fem grupper, hvor fire av
dem deltok i ulike treningsprotokoller i 12 uker, og en var
kontrollgruppe. En gruppe trente utholdenhet, en gruppe trente sirkel
styrketrening, en gruppe trente styrke også utholdenhet, og en gruppe
trente utholdenhet også styrke, dette trente de to ganger i uken.
Pretestene og post testene ble gjennomført på en løpebane og i et
laboratorium. Testene som ble gjennomført ute ble standardisert så godt
det lar seg gjøre, men det er en del utenomliggende faktorer man ikke
kan kontrollere. Laboratorium testene ble gjennomført i standardiserte
forhold. ~Det ble gjennomført et pre test før intervensjonen og en post
test etter intervensjonen. Det ble brukt en paret t-test til å
gjennomføre de statistiske analysene, og P \textless0,05

~

Lo et al.~(2011) gjennomførte en RCT studie. Gruppene bestod av totalt
34 friske og inaktive mannlige studenter med en snittalder på 20,4 år.
De ble tilfeldig fordelt i tre ulike grupper. En gruppe trente
utholdenhet 3 ganger i uka i 24 uker, en gruppe trente styrke 3 ganger i
uka i 24 uker, og en gruppe var kontrollgruppe. Etter de 24 ukene med
trening, hadde alle 24 uker uten trening. Det ble gjennomført tester før
intervensjonsperioden, etter 24 uker, og en post test etter 48 uker.
Prestasjonstestene ble gjennomført i standardiserte forhold med
standardiserte protokoller enten på tredemølle eller styrke maskiner.
Det samme ble de fysiologiske testene. De statistiske analysene ble
gjennomført ved en veis eller to veis ANOVA. Hvor P \textless{} 0,05.

~

Warburton et al.~(2004) gjennomførte en RCT studie. Gruppene bestod av
20 normalt aktive mannlige frivillige deltakere. De ble fordelt
tilfeldig i to forskjellige grupper, en gruppe som trente
intervalltrening og en gruppe som trente utholdenhetstrening med
kontinuerlig arbeid. Begge gruppene trente 3 dager i uka i 12 uker og
hadde minimum 24 t hvile mellom hver økt. Det ble gjennomført et pre
test og en post test før og etter intervensjonsperioden. Testene ble
gjennomført på en standardisert måte under standardiserte forhold. De
statistiske analysene som ble brukt var ANOVA, post hoc og lineær
regresjon, signifikantnivået ble satt til P \textless{} 0,05.

~

Alle studiene gjennomførte en RCT studie. Det er forskjellig antall
forsøkspersoner i hver studie, men alle gruppene bestod av minst 8
personer. Ingen av studiene sier noe om hvorvidt dette er et
representativt utvalg, men det er ikke noen forskjeller mellom gruppene
ved pretest. Ingen av studiene sier noe om hvordan forsøkspersonene ble
rekruttert, men alle forsøkspersonene har gitt samtykke til å være med,
så de deltar frivillig. Det er ikke skrevet noe om sample size i noen av
de fem studiene. I de aller fleste studiene ble det brukt ANOVA test for
å analysere resultatene, og signifikansnivået ble oppgitt i P-verdi. Kun
en av studiene hadde med VO\textsubscript{2}maks i hypotesen sin, men
alle oppgavene svarer på hvorvidt VO\textsubscript{2}maks blir påvirket
av intervensjonsperioden. Alle studiene svarer og på sin egen
problemstilling gjennom å sammenligne pre og post verdier, og analysere
endringen.

~

\hypertarget{resultat}{%
\section{Resultat}\label{resultat}}

~

Alle studiene svarer på sin egen problemstilling, og tester de
variablene som inngår i hypotesen eller problemstillingen. De tester og
andre variabler som for eksempel VO\textsubscript{2}maks, som i 4 av 5
studier ikke inngår i hypotesen. Dette er typisk for idrettsforskning,
hvor man gjerne har så store testbatterier at man og finner andre
resultater som man egentlig ikke leter etter. Disse blir gjerne brukt
videre i andre artikler eller brukt som grunnlaget for ny forskning.
Dette gjør og at man kan gjøre som det blir gjort i denne oppgaven, og
se på hvordan ulik intervensjon påvirker VO\textsubscript{2}maks, selv
om det ikke er en del av hovedfunnene til noen av studiene man bruker.

~

\hypertarget{interferens}{%
\section{Interferens}\label{interferens}}

~

Alle studiene konkluderte med at de hadde funnet ut at hypotesen var
delvis eller helt rett. Alle studiene konkluderte med at den gruppa som
hadde trent høyintensivt eller hadde trent styrke hadde den beste
helseeffekten, det var ikke alle de testede variablene man fikk
signifikante funn, men man fikk uten unntak størst økning i
VO\textsubscript{2}maks på den gruppa som hadde trent intervalltrening.
Det var kun Lo et al (2011) som konkluderte med hvilken praktisk
betydning funnene deres kunne ha for resten av befolkningen. Alle de
andre studiene konkluderte med om man hadde svart på hypotesen eller
problemstillingen sin.

\hypertarget{kilder}{%
\section{Kilder}\label{kilder}}

~

~

Chtara, M. (2005). Effects of intra-session concurrent endurance and
strength training sequence on aerobic performance and capacity.
\emph{British Journal of Sports Medicine}, \emph{39}(8), 555--560.
\url{https://doi.org/10.1136/bjsm.2004.015248}

Lo, M. S., Lin, L. L. C., Yao, W.-J., \& Ma, M.-C. (2011). Training and
Detraining Effects of the Resistance vs.~Endurance Program on Body
Composition, Body Size, and Physical Performance in Young Men.
\emph{Journal of Strength and Conditioning Research}, \emph{25}(8),
2246--2254. \url{https://doi.org/10.1519/JSC.0b013e3181e8a4be}

Nybo, L., Sundstrup, E., Jakobsen, M. D., Mohr, M., Hornstrup, T.,
Simonsen, L., Bülow, J., Randers, M. B., Nielsen, J. J., Aagaard, P., \&
Krustrup, P. (2010). High-Intensity Training versus Traditional Exercise
Interventions for Promoting Health. \emph{Medicine \& Science in Sports
\& Exercise}, \emph{42}(10), 1951--1958.
\url{https://doi.org/10.1249/MSS.0b013e3181d99203}

Trapp, E. G., Chisholm, D. J., Freund, J., \& Boutcher, S. H. (2008).
The effects of high-intensity intermittent exercise training on fat loss
and fasting insulin levels of young women. \emph{International Journal
of Obesity}, \emph{32}(4), 684--691.
\url{https://doi.org/10.1038/sj.ijo.0803781}

Warburton, D. E. R., Haykowsky, M. J., Quinney, H. A., Blackmore, D.,
Teo, K. K., Taylor, D. A., Mcgavock, J., \& Humen, D. P. (2004). Blood
Volume Expansion and Cardiorespiratory Function: Effects of Training
Modality: \emph{Medicine \& Science in Sports \& Exercise},
\emph{36}(6), 991--1000.
\url{https://doi.org/10.1249/01.MSS.0000128163.88298.CB}

\end{document}
